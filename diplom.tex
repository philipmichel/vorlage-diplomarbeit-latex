% Header mit Deklarationen
\documentclass[
	pdftex,%              PDFTex verwenden
	a4paper,%             A4 Papier
	oneside,%             Einseitig
	bibtotoc,%    		Literaturverzeichnis einf�gen bibtotocnumbered: nummeriert
	liststotoc,%		Verzeichnisse einbinden in toc
	idxtotoc,%            Index ins Verzeichnis einf�gen
	halfparskip,%        Europ�ischer Satz mit abstand zwischen Abs�tzen
	chapterprefix,%       Kapitel anschreiben als Kapitel
	headsepline,%         Linie nach Kopfzeile
	%footsepline,%         Linie vor Fusszeile
	%pointlessnumbers,%     Nummern ohne abschlie�enden Punkt
	12pt%                 Gr�ssere Schrift, besser lesbar am bildschrim
]{scrbook}

%
% Paket fuer Uebersetzungen ins Deutsche
%
\usepackage[ngerman]{babel}

%
% Pakete um Latin1 Zeichnensaetze verwenden zu koennen und die dazu
% passenden Schriften.
%
\usepackage[latin1]{inputenc}
\usepackage[T1]{fontenc}

%
% Paket fuer Quotes
%
\usepackage[babel,french=guillemets,german=swiss]{csquotes}

%
% Paket zum Erweitern der Tabelleneigenschaften
%
\usepackage{array}

%
% Paket fuer schoenere Tabellen
%
\usepackage{booktabs}

%
% Paket um Grafiken einbetten zu koennen
%
\usepackage{graphicx}

%
% Spezielle Schrift im Koma-Script setzen.
%
\setkomafont{sectioning}{\normalfont\bfseries}
\setkomafont{captionlabel}{\normalfont\bfseries} 
\setkomafont{pagehead}{\normalfont\bfseries} % Kopfzeilenschrift
\setkomafont{descriptionlabel}{\normalfont\bfseries}

%
% Zeilenumbruch bei Bildbeschreibungen.
%
\setcapindent{1em}

%
% Kopf und Fu�zeilen
%
\usepackage{scrpage2}
\pagestyle{scrheadings}
% Inhalt bis Section rechts und Chapter links
\automark[section]{chapter}
% Mitte: leer
\chead{}

%
% mathematische symbole aus dem AMS Paket.
%
\usepackage{amsmath}
\usepackage{amssymb}

%
% Type 1 Fonts f�r bessere darstellung in PDF verwenden.
%
%\usepackage{mathptmx}           % Times + passende Mathefonts
%\usepackage[scaled=.92]{helvet} % skalierte Helvetica als \sfdefault
\usepackage{courier}            % Courier als \ttdefault

%
% Paket um Textteile drehen zu k�nnen
%
\usepackage{rotating}

%
% Paket fuer Farben im PDF
%
\usepackage{color}

%
% Paket fuer Links innerhalb des PDF Dokuments
%
\definecolor{LinkColor}{rgb}{0,0,0.5}
\usepackage[%
	pdftitle={Titel},% Titel der Diplomarbeit
	pdfauthor={Autor},% Autor(en)
	pdfcreator={LaTeX, LaTeX with hyperref and KOMA-Script},% Genutzte Programme
	pdfsubject={Betreff}, % Betreff
	pdfkeywords={Keywords}]{hyperref} % Keywords halt :-)
	
\hypersetup{colorlinks=true,% Definition der Links im PDF File
	linkcolor=LinkColor,%
	citecolor=LinkColor,%
	filecolor=LinkColor,%
	menucolor=LinkColor,%
	pagecolor=LinkColor,%
	urlcolor=LinkColor}

%
% Paket um LIstings sauber zu formatieren.
%
\usepackage[savemem]{listings}
\lstloadlanguages{TeX}

%
% Listing Definationen f�r PHP Code
%
\definecolor{lbcolor}{rgb}{0.85,0.85,0.85}
\lstset{language=[LaTeX]TeX,
	numbers=left,
	stepnumber=1,
	numbersep=5pt,
	numberstyle=\tiny,
	breaklines=true,
	captionpos=b,				% caption-position bottom
	breakautoindent=true,
	postbreak=\space,
	tabsize=2,
	basicstyle=\ttfamily\footnotesize,
	showspaces=false,
	showstringspaces=false,
	extendedchars=true,
	backgroundcolor=\color{lbcolor}}
%
% ---------------------------------------------------------------------------
%


%
% Neue Umgebungen
%
\newenvironment{ListChanges}%
	{\begin{list}{$\diamondsuit$}{}}%
	{\end{list}}

%
% aller Bilder werden im Unterverzeichnis figures gesucht:
%
\graphicspath{{bilder/}}

%
% Literaturverzeichnis-Stil
%
\bibliographystyle{plain}

%
% Anfuehrungsstriche mithilfe von \textss{-anzufuehrendes-}
%
\newcommand{\textss}[1]{"`#1"'}

%
% Strukturiertiefe bis subsubsection{} moeglich
%
\setcounter{secnumdepth}{3}

%
% Dargestellte Strukturiertiefe im Inhaltsverzeichnis
%
\setcounter{tocdepth}{3}

%
% Zeilenabstand wird um den Faktor 1.5 veraendert
%
%\renewcommand{\baselinestretch}{1.5}

%
% Abkuerzungsverzeichnis
%
\usepackage{nomencl}
% Befehl umbenennen in abk
\let\abk\nomenclature
% Deutsche Ueberschrift
\renewcommand{\nomname}{Abk\"urzungsverzeichnis}
% Punkte zw. Abkuerzung und Erklaerung
\setlength{\nomlabelwidth}{.20\hsize}
\renewcommand{\nomlabel}[1]{#1 \dotfill}
% Zeilenabstaende verkleinern
\setlength{\nomitemsep}{-\parsep}
\makenomenclature

\usepackage{eso-pic}
\newcommand\BackgroundPic{%
\put(0,0){%
\parbox[b][\paperheight]{\paperwidth}{%
\vfill
\centering
\includegraphics[width=0.8\paperwidth,height=0.8\paperheight,%
keepaspectratio]{bilder/titelbild.png}%
\vfill
}}}

%
% Kapitelueberschriften werden in eine Zeile geschrieben.
%
% normal: 
% Kapitel 1.
% Einleitung
%
% wenn hier auskommentiert:
% Kapitel 1: Einleitung
%
%\usepackage{titlesec}
%\titleformat{\chapter}[hang] 
%{\normalfont\huge\bfseries}{\chaptertitlename\ \thechapter:}{1em}{} 



\begin{document}
% Hintergrundbild fuer Titelseite einfuegen
% nur bei Titelseite von HTL-Shkoder einkommentieren
% nicht bei Titleseitesimpel
\AddToShipoutPicture*{\BackgroundPic}

% Roemische Nummerierung fuer Sonderseiten, wie Kurzfassung, Abstract, Verzeichnisse und Anhang
\pagenumbering{Roman}

% Titelblatt HTL-Shkoder
% Die Titelseite

\newcommand{\trtitle}{Titel}
\newcommand{\trtype}{Diplomarbeit}
\newcommand{\trort}{Shkoder}
\newcommand{\trbetreuer}{Titel Betreuer}
\newcommand{\trfachgebiet}{SEW, INSY, NWTK, ...}
\newcommand{\trdate}{\today}

\thispagestyle{empty}

% Kopfzeile mit Logos.
% Eventuell die \hspace{} 
%\begin{tabular}{lcr}
% \includegraphics[scale=0.8]{bildname} & % dein_unilogo.jpg/.eps im Verzeichnis "bilder" ablegen
%  \hspace{2cm} \truni \hspace{2cm} &
 % \includegraphics[scale=0.8]{logoname} % dein_fglogo.jpg/.eps im Verzeichnis "bilder" ablegen, Fachgebietslogo
 % \\
%\end{tabular}

\rule{\textwidth}{0.4pt}

\vspace{2.5cm}
\begin{center}
  \textbf{\LARGE \trtitle}
\end{center}
\vspace{2cm}

\begin{center}
  \textbf{\trtype} \\
  am Fachgebiet \trfachgebiet \\
  \vspace{2.5cm}
  vorgelegt von \\
  \textbf{Vorname Nachname}\\
   \textbf{Vorname Nachname}\\
    \textbf{Vorname Nachname}
\end{center}

\vspace{1cm}


\begin{center}
\begin{tabular}{ll}
Betreuer: & \trbetreuer \\
\end{tabular}
\end{center}

\vfill


\rule{\textwidth}{0.4pt}

% einfaches Titelblatt
%% Die Titelseite
% Im folgenden kommen ein paar Variablen, die auszuf�llen sind
% Bisher steht dort nur Musterinhalt
% Au�erdem m�ssen zei Dateien erstellt werden, Bild/Logo/Emblem des Fachgebietes
% sowie der Universit�t

\newcommand{\trtitle}{Titel der Diplomarbeit}
\newcommand{\trort}{Shkoder}
\newcommand{\trbetreuer}{Titel Betreuer}
\newcommand{\trdate}{\today}
\newcommand{\trnumber}{15.xx}
\newcommand{\trclass}{5X}
\newcommand{\trschuelereins}{Sch\"uler1}
\newcommand{\trschuelerzwei}{Sch\"uler2}
\newcommand{\trschuelerdrei}{Sch\"uler3}
\newcommand{\trsauftraggeber}{Herr Max Mustermann oder Firma}
\newcommand{\trbetreuereins}{Lehrer1}
\newcommand{\trbetreuerzwei}{Lehrer2}
\newcommand{\trbetreuerdrei}{Lehrer3}

\thispagestyle{empty}

\begin{center}
  \Large H\"ohere technische Schule f\"ur Informationstechnologie
\end{center}

\begin{center}
  \includegraphics[scale=0.2]{htllogo} \\
  \textbf{\LARGE \trtitle}
\end{center}
\vspace{1cm}

\begin{flushleft}
\textbf{\LARGE Diplomarbeit Nr. \trnumber} \\
\LARGE Klasse \trclass{}, Schuljahr 2015/16

\begin{table}[htbp]
\Large
\begin{tabular}{cc}
   Ausgef\"uhrt von: & \trschuelereins \\ 
   & \trschuelerzwei \\ 
   & \trschuelerdrei \\ 
 \end{tabular}
\end{table}
\end{flushleft}


\large Projektbetreuer1: \trbetreuereins \\
\large Projektbetreuer2: \trbetreuerzwei \\
\large Projektbetreuer3: \trbetreuerdrei \\
\newline
\large \trort{}, \today

\vfill

% Eidesstattliche Erklaerung
% Die eidesstattliche Erklaerung mit Unterschrift
\chapter*{Eidesstattliche Erkl\"arung}

Wir versichern, dass wir die vorstehende Arbeit selbstst\"andig und ohne fremde Hilfe angefertigt haben. Wir haben uns keiner anderen als der im beigef\"ugten Quellenverzeichnis angegebenen Hilfsmittel bedient. Alle Stellen, die w\"ortlich oder sinngem\"a\ss aus Ver\"offentlichungen entnommen wurden, sind als solche kenntlich gemacht.


\vspace{2.5cm}

\hspace{2cm} Ort, Datum \hfill Unterschrift \hspace{2cm}

\vspace{2cm}

\hspace{2cm} Ort, Datum \hfill Unterschrift \hspace{2cm}

\vspace{2cm}

\hspace{2cm} Ort, Datum \hfill Unterschrift \hspace{2cm}

\vspace{2cm}

\begin{center}
S\"amtliche in dieser Diplomarbeit verwendeten personenbezogenen Bezeichnungen sind geschlechtsneutral zu verstehen.
\end{center}

% Kurzfassung
%
% neue Seite
%
\newpage

%
% Ueberschrift Kurzfassung
%
\section*{Kurzfassung}
Text\\
TextTextTextTextTextTextTextTextTextTextTextTextText TextTextTextTextTextTextTextTextTextTextTextTextText TextTextTextTextTextTextTextTextTextTextTextTextText TextTextTextTextTextTextTextTextTextTextTextTextText TextTextTextTextTextTextTextTextTextTextTextTextText

\color{red} 
Die Kurzfassung fasst die Arbeit in max. 200 Worten zusammen. 
\begin{itemize}
 \item Was ist das Problem / die Aufgabenstellung / Fragestellung gewesen?
 \item Was ist das Ziel der Diplomarbeit?
 \item Theoretischer Hintergrund
 \item Methodik(en)
 \item Was ist das (Kern-)Ergebnis?
\end{itemize}
Die Kurzfassung ersetzt gewisserma{\ss}en das \"Uberfliegen des eigentlichen Textes!
Die Kurzfassung soll einen \"Uberblick \"uber die Arbeit geben, sowie den \glqq{}roten Faden\grqq{} und die wichtigsten Details f\"ur den Leser liefern. Sie muss informativ sein, unabh\"angig ob sie alleine oder zusammen mit der Arbeit gelesen wird.
Eine gute Kurzfassung hat selten mehr als 100 - 200 Worte und fasst kurz und pr\"agnant die Thematik, das Ziel der Arbeit, die verwendeten Methoden und (Kern-)Ergebnisse bzw. Erkenntnisse zusammen. Die Kurzfassung ist der zuletzt erstellte Teil der Arbeit.

\color{black} 


% Abstract
%
% neue Seite
%
\newpage

%
% Ueberschrift Abstract
%
\section*{Abstract}
Text\\
TextTextTextTextTextTextTextTextTextTextTextTextText TextTextTextTextTextTextTextTextTextTextTextTextText TextTextTextTextTextTextTextTextTextTextTextTextText TextTextTextTextTextTextTextTextTextTextTextTextText TextTextTextTextTextTextTextTextTextTextTextTextText

\color{red} 
Ein deskriptives Abstract beschreibt folgende Aspekte der Arbeit: den Hintergrund und die Motivation der Arbeit, die Problemstellung, den Umfang sowie die Grenzen der Arbeit und die zur L\"osung verwendeten Methoden. Ein Abstract sollte maximal100 Worte umfassen und den Leser motivieren, sich mit den im Hauptteil beschriebenen Resultaten, Empfehlungen und Schl\"ussen auseinander zu setzen. Das Abstract wird von Grund auf in englischer Sprache abgefasst und nicht aus dem Deutschen \"ubersetzt. NACH dem Verfassen der englischen Version muss das Abstract auf Albanisch und auf Deutsch \"ubersetzt werden. \"Ahnlich wie die Kurzfassung wird das Abstract \"ublicherweise zum Schluss geschrieben.
\color{black} 


% Danksagung
%
% neue Seite
%
\newpage

%
% Ueberschrift Abstract
%
\section*{Danksagung}
Text\\
TextTextTextTextTextTextTextTextTextTextTextTextText TextTextTextTextTextTextTextTextTextTextTextTextText TextTextTextTextTextTextTextTextTextTextTextTextText TextTextTextTextTextTextTextTextTextTextTextTextText TextTextTextTextTextTextTextTextTextTextTextTextText

\vspace{4cm}

Wer m\"ochte kann eine Danksagung verfassen!

% Verzeichnisse
% Kopfzeile links Kapitel, rechts leer
\ihead{\leftmark}
\ohead{}

% Inhaltsverzeichnis, Abbildungsverzeichnis, Tabellenverzeichnis
\input{extras/verzeichnisse}

% Merke mir die roemische Seitenzahl in 'roemisch' und setzte Nummerierung 
% auf arabisch fuer die eigentlichen Kapitel
\newpage
\newcounter{roemisch}
\setcounter{roemisch}{\value{page}}
\pagenumbering{arabic}

% Die einzelnen Kapitel
% Kopfzeile: links Kapitel, rechts Sektion
%\ihead{\leftmark}
%\ohead{\rightmark}
%\ihead{\parbox[t][2\baselineskip][t]{\dimexpr\linewidth-2em\relax}{\leftmark}}
%\ohead{\parbox[t][2\baselineskip][t]{20em}{\raggedleft\rightmark}}
\chapter{Allgemeines}
\section{Idee, Thema, Aufgabenstellung}
Text Text Text Text Text Text Text Text Text Text Text Text Text Text Text Text Text Text

\chapter{Planung}
\section{Projektziele}
Text Text Text Text Text Text Text Text Text Text Text Text Text Text Text Text Text Text
\section{Projektplanung}
Text Text Text Text Text Text Text Text Text Text Text Text Text Text Text Text Text Text
\section{Projektmanagementmethode}
Text Text Text Text Text Text Text Text Text Text Text Text Text Text Text Text Text Text

\chapter{Dokumentation des Projekverlaufs}
\section{Allgemeine Beschreibungen}
Text Text Text Text Text Text Text Text Text Text Text Text Text Text Text Text Text Text
\section{Technische L\"osungen}
Text Text Text Text Text Text Text Text Text Text Text Text Text Text Text Text Text Text
\section{Beschreibungen des Arbeitsverlaufs}
Text Text Text Text Text Text Text Text Text Text Text Text Text Text Text Text Text Text
\section{.. und so weiter}
Text Text Text Text Text Text Text Text Text Text Text Text Text Text Text Text Text Text
\chapter{Viertes Kapitel}
\section{Listen und Aufz\"ahlungen}

Hier mal eine Auflistung von Elementen
\begin{itemize}
 \item erstes Element
 \item zweites Element
 \item noch ein Element
\end{itemize}

Hier mal eine Aufz\"ahlung
\begin{enumerate}
 \item erster Punkt
 \item noch ein Punkt
 \item letzter Punkt
\end{enumerate}

\section{Und n\"achster Abschnitt etwas l\"anger als vorher es war}
Eine neue Seite, um auchmal die Kopfzeile zu sehen, da sie auf Seiten mit Kapitelanfang nicht erscheinen. Eine Abk\"urzung ist z.B. etc.\abk{etc.}{et cetera}.

\section{Eine Tabelle}

Hier eine Tabelle:
\begin{table}[htbp]
\centering
\begin{tabular}{l|l|l|l}
SpalteA & SpalteB & SpalteC & SpalteD \\
\midrule
InhaltA1 & InhaltB1 & InhaltC1 & InhaltD1 \\
InhaltA2 & InhaltB2 & InhaltC2 & InhaltD2 \\
InhaltA3 & InhaltB3 & InhaltC3 & InhaltD3
\end{tabular}
\caption{Beispiel einer Tabelle}
\label{tab:tabelle1}
\end{table}

Wie man in der Tabelle~\ref{tab:tabelle1} sehen kann ...

% Neue Seite
\section{Zitieren}
Zitierregeln sind der Word-Version der Diplomarbeitsvorlage zu entnehmen! \\
In der Datei \textit{bib.bib} im Verzeichnis \textit{Literatur} sind neue Eintr\"age von Literatur hinzuzuf\"ugen. Das Format kann in der \LaTeX{} Dokumentation nachgesehen werden. \\
N\"utzliche Links zum automatischen Erstellen von BibTeX Eintr\"agen:
\begin{itemize}
	\item http://truben.no/latex/bibtex/
	\item http://www.ottobib.com
	\item http://www.literatur-generator.de
	\item https://scholar.google.de
\end{itemize}
Damit das Literaturverzeichnis erstellt wird, muss auch mit dem Befehl BibTeX kompiliert werden! \\
\subsection{Beispiele}
Die exakte Herleitung kann in \cite{Prager1961} nachgelesen werden. \\
Hier ein Zitat etwas einger\"uckt:
\begin{quote}
Das Programm TeX wurde von Donald E. Knuth, Professor an der Stanford University, entwickelt. Leslie Lamport entwickelte Anfang der 1980er Jahre darauf aufbauend LaTeX, eine Sammlung von TeX-Makros. Der Name ist eine Abk\"urzung für Lamport TeX \cite{latextug}.
\end{quote}

\section{Bilder und Referenzen}
\begin{figure}[htbp]
\centering
\includegraphics[width=0.7\textwidth]{abbildungsdatei} % Datei in "bilder/" bei LaTeX: eps, bei PDFLaTeX: jpg (o.�.) 
\caption{Titel der Abbildung} 
\label{fig:bild1}
\end{figure}

In der Abbildung~\ref{fig:bild1}\footnote{vgl. Zitat A\cite{referenzA}} % referenzA muss im Verzeichnis literatur in der Datei bib.bib enthalten sein
ist zu sehen, dass ...

\section{Source Code einbinden}
Hier eine Einbindung von Source Code, funktioniert mit allen g\"angigen Programmiersprachen:
\lstinputlisting{./src/HelloWorld.java}


% Setze Numerierung wieder auf roemisch zuruek und setzte von oben fort
% Wert ist demnach der von 'roemisch'
\newpage
\pagenumbering{Roman}
\setcounter{page}{\value{roemisch}}

% Literaturverzeichnis
\bibliographystyle{plain}
\bibliography{literatur/bib}

% Appendix, falls vorhanden
\appendix
\input{extras/anhang}

\end{document}
