% Header mit Deklarationen
%
% Paket fuer Uebersetzungen ins Deutsche
%
\usepackage[ngerman]{babel}

%
% Pakete um Latin1 Zeichnensaetze verwenden zu koennen und die dazu
% passenden Schriften.
%
\usepackage[latin1]{inputenc} % fuer OS X
% \usepackage[utf8]{inputenc} % fuer Windows
\usepackage[T1]{fontenc}

%
% Paket fuer Quotes
%
\usepackage[babel,french=guillemets,german=swiss]{csquotes}

%
% Paket zum Erweitern der Tabelleneigenschaften
%
\usepackage{array}

%
% Paket fuer schoenere Tabellen
%
\usepackage{booktabs}

%
% Paket um Grafiken einbetten zu koennen
%
\usepackage{graphicx}

%
% Spezielle Schrift im Koma-Script setzen.
%
\setkomafont{sectioning}{\normalfont\bfseries}
\setkomafont{captionlabel}{\normalfont\bfseries} 
\setkomafont{pagehead}{\normalfont\bfseries} % Kopfzeilenschrift
\setkomafont{descriptionlabel}{\normalfont\bfseries}

%
% Zeilenumbruch bei Bildbeschreibungen.
%
\setcapindent{1em}

%
% Kopf und Fu�zeilen
%
\usepackage{scrpage2}
\pagestyle{scrheadings}
% Inhalt bis Section rechts und Chapter links
\automark[section]{chapter}
% Mitte: leer
\chead{}

%
% mathematische symbole aus dem AMS Paket.
%
\usepackage{amsmath}
\usepackage{amssymb}

%
% Type 1 Fonts f�r bessere darstellung in PDF verwenden.
%
%\usepackage{mathptmx}           % Times + passende Mathefonts
%\usepackage[scaled=.92]{helvet} % skalierte Helvetica als \sfdefault
\usepackage{courier}            % Courier als \ttdefault

%
% Paket um Textteile drehen zu k�nnen
%
\usepackage{rotating}

%
% Paket fuer Farben im PDF
%
\usepackage{color}

%
% Paket fuer Links innerhalb des PDF Dokuments
%
\definecolor{LinkColor}{rgb}{0,0,0.5}
\usepackage[%
	pdftitle={Titel},% Titel der Diplomarbeit
	pdfauthor={Autor},% Autor(en)
	pdfcreator={LaTeX, LaTeX with hyperref and KOMA-Script},% Genutzte Programme
	pdfsubject={Betreff}, % Betreff
	pdfkeywords={Keywords}]{hyperref} % Keywords halt :-)
	
\hypersetup{colorlinks=true,% Definition der Links im PDF File
	linkcolor=LinkColor,%
	citecolor=LinkColor,%
	filecolor=LinkColor,%
	menucolor=LinkColor,%
	pagecolor=LinkColor,%
	urlcolor=LinkColor}

%
% Paket um LIstings sauber zu formatieren.
%
\usepackage[savemem]{listings}
\lstloadlanguages{TeX}

%
% Listing Definationen f�r PHP Code
%
\definecolor{lbcolor}{rgb}{0.85,0.85,0.85}
\lstset{language=[LaTeX]TeX,
	numbers=left,
	stepnumber=1,
	numbersep=5pt,
	numberstyle=\tiny,
	breaklines=true,
	captionpos=b,				% caption-position bottom
	breakautoindent=true,
	postbreak=\space,
	tabsize=2,
	basicstyle=\ttfamily\footnotesize,
	showspaces=false,
	showstringspaces=false,
	extendedchars=true,
	backgroundcolor=\color{lbcolor}}
%
% ---------------------------------------------------------------------------
%


%
% Neue Umgebungen
%
\newenvironment{ListChanges}%
	{\begin{list}{$\diamondsuit$}{}}%
	{\end{list}}

%
% aller Bilder werden im Unterverzeichnis figures gesucht:
%
\graphicspath{{bilder/}}

%
% Literaturverzeichnis-Stil
%
\bibliographystyle{plain}

%
% Anfuehrungsstriche mithilfe von \textss{-anzufuehrendes-}
%
\newcommand{\textss}[1]{"`#1"'}

%
% Strukturiertiefe bis subsubsection{} moeglich
%
\setcounter{secnumdepth}{3}

%
% Dargestellte Strukturiertiefe im Inhaltsverzeichnis
%
\setcounter{tocdepth}{3}

%
% Zeilenabstand wird um den Faktor 1.5 veraendert
%
%\renewcommand{\baselinestretch}{1.5}

%
% Abkuerzungsverzeichnis
%
\usepackage{nomencl}
% Befehl umbenennen in abk
\let\abk\nomenclature
% Deutsche Ueberschrift
\renewcommand{\nomname}{Abk\"urzungsverzeichnis}
% Punkte zw. Abkuerzung und Erklaerung
\setlength{\nomlabelwidth}{.20\hsize}
\renewcommand{\nomlabel}[1]{#1 \dotfill}
% Zeilenabstaende verkleinern
\setlength{\nomitemsep}{-\parsep}
\makenomenclature

\usepackage{eso-pic}
\newcommand\BackgroundPic{%
\put(0,0){%
\parbox[b][\paperheight]{\paperwidth}{%
\vfill
\centering
\includegraphics[width=0.8\paperwidth,height=0.8\paperheight,%
keepaspectratio]{bilder/titelbild.png}%
\vfill
}}}

%
% Kapitelueberschriften werden in eine Zeile geschrieben.
%
% normal: 
% Kapitel 1.
% Einleitung
%
% wenn hier auskommentiert:
% Kapitel 1: Einleitung
%
%\usepackage{titlesec}
%\titleformat{\chapter}[hang] 
%{\normalfont\huge\bfseries}{\chaptertitlename\ \thechapter:}{1em}{} 



\begin{document}
% Hintergrundbild fuer Titelseite einfuegen
% nur bei Titelseite von HTL-Shkoder einkommentieren
% nicht bei Titleseitesimpel
\AddToShipoutPicture*{\BackgroundPic}

% Roemische Nummerierung fuer Sonderseiten, wie Kurzfassung, Abstract, Verzeichnisse und Anhang
\pagenumbering{Roman}

% Titelblatt HTL-Shkoder
% Die Titelseite

\newcommand{\trtitle}{Titel der Diplomarbeit}
\newcommand{\trort}{Shkoder}
\newcommand{\trbetreuer}{Titel Betreuer}
\newcommand{\trfachgebiet}{SEW, INSY, NWTK, ...}
\newcommand{\trdate}{\today}
\newcommand{\trnumber}{15.xx}
\newcommand{\trclass}{5X}
\newcommand{\trschuelereins}{Sch\"uler1}
\newcommand{\trschuelerzwei}{Sch\"uler2}
\newcommand{\trschuelerdrei}{Sch\"uler3}
\newcommand{\trsauftraggeber}{Herr Max Mustermann oder Firma}
\newcommand{\trbetreuereins}{Lehrer1}
\newcommand{\trbetreuerzwei}{Lehrer2}
\newcommand{\trbetreuerdrei}{Lehrer3}

\thispagestyle{empty}

\vspace{0.1cm}
\begin{flushleft}
\textbf{\LARGE Diplomarbeit Nr. \trnumber} \\
\LARGE Klasse \trclass{}, Schuljahr 2015/16

\vspace{9cm}
\textbf{\LARGE \trtitle}

\vspace{0.5cm}
\begin{table}[htbp]
\Large
\begin{tabular}{cl}
   Ausgef\"uhrt von: & \trschuelereins \\ 
   & \trschuelerzwei \\ 
   & \trschuelerdrei \\ 
 \end{tabular}
\end{table}
\end{flushleft}

\vspace{0.4cm}
\Large Auftraggeber: \trsauftraggeber

\vspace{0.4cm}
\large Projektbetreuer1: \trbetreuereins \\
\large Projektbetreuer2: \trbetreuerzwei \\
\large Projektbetreuer3: \trbetreuerdrei \\
\newline
\large \trort{}, \today 

\vfill

% einfaches Titelblatt
%% Die Titelseite
% Im folgenden kommen ein paar Variablen, die auszuf�llen sind
% Bisher steht dort nur Musterinhalt
% Au�erdem m�ssen zei Dateien erstellt werden, Bild/Logo/Emblem des Fachgebietes
% sowie der Universit�t

\newcommand{\trtitle}{Titel der Diplomarbeit}
\newcommand{\trort}{Shkoder}
\newcommand{\trbetreuer}{Titel Betreuer}
\newcommand{\trdate}{\today}
\newcommand{\trnumber}{16.xx}
\newcommand{\trclass}{5X}
\newcommand{\trschuelereins}{Sch\"uler1}
\newcommand{\trschuelerzwei}{Sch\"uler2}
\newcommand{\trschuelerdrei}{Sch\"uler3}
\newcommand{\trschuelervier}{Sch\"uler4}
\newcommand{\trschuelerfuenf}{Sch\"uler5}
\newcommand{\trsauftraggeber}{Herr Max Mustermann oder Firma}
\newcommand{\trbetreuereins}{Lehrer1}
\newcommand{\trbetreuerzwei}{Lehrer2}
\newcommand{\trbetreuerdrei}{Lehrer3}

\thispagestyle{empty}

\begin{center}
  \large H\"ohere technische Schule f\"ur Informationstechnologie \\
  \large Shkolla e mesme profesionale private p\"er teknologji informacioni \\
  \Huge \"Osterreichiche Schule \glqq Peter Mahringer\grqq \\
  \Large Shkolla Austriake Shkod\"er
\end{center}

\begin{center}
  \includegraphics[scale=0.2]{htllogo} \\
  \textbf{\LARGE \trtitle}
\end{center}
\vspace{1cm}

\begin{flushleft}
\textbf{\LARGE Diplomarbeit Nr. \trnumber} \\
\LARGE Klasse \trclass{}, Schuljahr 2016/17

\begin{table}[htbp]
\Large
\begin{tabular}{cl}
   Ausgef\"uhrt von: & \trschuelereins \\ 
   & \trschuelerzwei \\ 
   & \trschuelerdrei \\ 
   & \trschuelervier \\ 
   & \trschuelerfuenf \\ 
 \end{tabular}
\end{table}
\end{flushleft}


\large Projektbetreuer1: \trbetreuereins \\
\large Projektbetreuer2: \trbetreuerzwei \\
\large Projektbetreuer3: \trbetreuerdrei \\
\newline
\large \trort{}, \today

\vfill

% Eidesstattliche Erklaerung
% Die eidesstattliche Erklaerung mit Unterschrift
\chapter*{Eidesstattliche Erkl\"arung}

Wir versichern, dass wir die vorliegende Arbeit selbstst\"andig und ohne fremde Hilfe angefertigt haben. Wir haben uns keiner anderen als der im beigef\"ugten Quellenverzeichnis angegebenen Hilfsmittel bedient. Alle Stellen, die w\"ortlich oder sinngem\"a\ss aus Ver\"offentlichungen entnommen wurden, sind als solche kenntlich gemacht.


\vspace{2.5cm}

\hspace{2cm} Ort, Datum \hfill Unterschrift \hspace{2cm}

\vspace{2cm}

\hspace{2cm} Ort, Datum \hfill Unterschrift \hspace{2cm}

\vspace{2cm}

\hspace{2cm} Ort, Datum \hfill Unterschrift \hspace{2cm}

\vspace{2cm}

\begin{center}
S\"amtliche in dieser Diplomarbeit verwendeten personenbezogenen Bezeichnungen sind geschlechtsneutral zu verstehen.
\end{center}

% Kurzfassung
\input{extras/kurzfassung}

% Abstract
\input{extras/abstract}

% Danksagung
\input{extras/danksagung}

% Verzeichnisse
% Kopfzeile links Kapitel, rechts leer
\ihead{\leftmark}
\ohead{}

% Inhaltsverzeichnis, Abbildungsverzeichnis, Tabellenverzeichnis
\input{extras/verzeichnisse}

% Merke mir die roemische Seitenzahl in 'roemisch' und setzte Nummerierung 
% auf arabisch fuer die eigentlichen Kapitel
\newpage
\newcounter{roemisch}
\setcounter{roemisch}{\value{page}}
\pagenumbering{arabic}

% Die einzelnen Kapitel
% Kopfzeile: links Kapitel, rechts Sektion
%\ihead{\leftmark}
%\ohead{\rightmark}
%\ihead{\parbox[t][2\baselineskip][t]{\dimexpr\linewidth-2em\relax}{\leftmark}}
%\ohead{\parbox[t][2\baselineskip][t]{20em}{\raggedleft\rightmark}}
\input{kapitel/kapitel1}
\input{kapitel/kapitel2}
\chapter{Dokumentation des Projektverlaufs}
\section{Allgemeine Beschreibungen}
Text Text Text Text Text Text Text Text Text Text Text Text Text Text Text Text Text Text
\section{Technische L\"osungen}
Text Text Text Text Text Text Text Text Text Text Text Text Text Text Text Text Text Text
\section{Beschreibungen des Arbeitsverlaufs}
Text Text Text Text Text Text Text Text Text Text Text Text Text Text Text Text Text Text
\section{.. und so weiter}
Text Text Text Text Text Text Text Text Text Text Text Text Text Text Text Text Text Text
\chapter{Viertes Kapitel}
\section{Listen und Aufz\"ahlungen}

Hier mal eine Auflistung von Elementen
\begin{itemize}
 \item erstes Element
 \item zweites Element
 \item noch ein Element
\end{itemize}

Hier mal eine Aufz\"ahlung
\begin{enumerate}
 \item erster Punkt
 \item noch ein Punkt
 \item letzter Punkt
\end{enumerate}

\section{Und n\"achster Abschnitt etwas l\"anger als vorher es war}
Eine neue Seite, um auchmal die Kopfzeile zu sehen, da sie auf Seiten mit Kapitelanfang nicht erscheinen. Eine Abk\"urzung ist z.B. etc.\abk{etc.}{et cetera}.

\section{Eine Tabelle}

Hier eine Tabelle:
\begin{table}[htbp]
\centering
\begin{tabular}{l|l|l|l}
SpalteA & SpalteB & SpalteC & SpalteD \\
\midrule
InhaltA1 & InhaltB1 & InhaltC1 & InhaltD1 \\
InhaltA2 & InhaltB2 & InhaltC2 & InhaltD2 \\
InhaltA3 & InhaltB3 & InhaltC3 & InhaltD3
\end{tabular}
\caption{Beispiel einer Tabelle}
\label{tab:tabelle1}
\end{table}

Wie man in der Tabelle~\ref{tab:tabelle1} sehen kann ...

% Neue Seite
\section{Zitieren und Literaturverzeichnis erzeugen}
Zitierregeln sind der Word-Version der Diplomarbeitsvorlage zu entnehmen! \\
In der Datei \textit{bib.bib} im Verzeichnis \textit{Literatur} sind neue Eintr\"age von Literatur hinzuzuf\"ugen. Das Format kann in der \LaTeX{} Dokumentation nachgesehen werden. \\
N\"utzliche Links zum automatischen Erstellen von BibTeX Eintr\"agen:
\begin{itemize}
	\item \url{http://truben.no/latex/bibtex/}
	\item \url{http://www.ottobib.com}
	\item \url{http://www.literatur-generator.de}
	\item \url{https://scholar.google.de}
\end{itemize}

\begin{center}
\Large{!!! \\Damit das Literaturverzeichnis erstellt wird, muss auch mit dem \textbf{Befehl BibTeX} kompiliert werden.\\ !!!}
\end{center}


\subsection{Beispiele}
Die exakte Herleitung kann in \cite{Prager1961} nachgelesen werden. \\
Hier ein Zitat etwas einger\"uckt:
\begin{quote}
Das Programm TeX wurde von Donald E. Knuth, Professor an der Stanford University, entwickelt. Leslie Lamport entwickelte Anfang der 1980er Jahre darauf aufbauend LaTeX, eine Sammlung von TeX-Makros. Der Name ist eine Abk\"urzung f\"ur Lamport TeX \cite{latextug}.
\end{quote}

\section{Bilder und Referenzen}
\begin{figure}[htbp]
\centering
\includegraphics[width=0.7\textwidth]{abbildungsdatei} % Datei in "bilder/" bei LaTeX: eps, bei PDFLaTeX: jpg (o.�.) 
\caption{Titel der Abbildung} 
\label{fig:bild1}
\end{figure}

In der Abbildung~\ref{fig:bild1}\footnote{vgl. Zitat A\cite{referenzA}} % referenzA muss im Verzeichnis literatur in der Datei bib.bib enthalten sein
ist zu sehen, dass ...

\section{Formeln}
\label{sec:formeln}

\subsection{Albert Einstein}
Keine Formel der modernen Physik ist in der allgemeinen \"Offentlichkeit wohl so bekannt wie die Einsteinsche Formel:
\begin{equation*}
  E = mc^2
\end{equation*}.

\subsection{Allgemeine quadratische Gleichung}
\begin{equation*}
  ax^2 + bx + c = 0
\end{equation*}

\begin{equation*}
  x_{1/2} = \frac{-b \pm \sqrt{b^2-4ac}}{2a}
\end{equation*}

\subsection{Intergral}
\begin{equation}
  \label{eq:1}
  \int\limits_{a}^{b} x^{2} \, dx = \frac{ b^{3} - a^{3} }{3}
\end{equation}
Siehe Beispiel \eqref{eq:1} \\
Siehe Abschnitt \ref{sec:formeln}

\section{Source Code einbinden}
\subsection{Datei}
Hier eine Einbindung von Source Code in Form einer Datei (funktioniert mit allen g\"angigen Programmiersprachen):
\lstinputlisting[label=lst:helloworldjava,caption=Hello World in Java]{./src/HelloWorld.java}
Die Ausgabe am Bildschirm von Hello World in Java ist in \ref{lst:helloworldjava} zu sehen.
\subsection{Inline}
Dieser Source Code ist direkt in \LaTeX{} eingegeben:
\begin{lstlisting}[caption=Hello World in C, label=lst:helloworldc]
#include <stdio.h>

/* Block
 * comment */
 
int main()
{
    // Line comment.
    printf("Hello World\n");
 
    return 0;
}
\end{lstlisting}
Die Ausgabe am Bildschirm von Hello World in Java ist in \ref{lst:helloworldc} zu sehen.

% Setze Numerierung wieder auf roemisch zuruek und setzte von oben fort
% Wert ist demnach der von 'roemisch'
\newpage
\pagenumbering{Roman}
\setcounter{page}{\value{roemisch}}

% Literaturverzeichnis
\bibliographystyle{plain}
\bibliography{literatur/bib}

% Appendix, falls vorhanden
\appendix
\input{extras/anhang}

\end{document}
