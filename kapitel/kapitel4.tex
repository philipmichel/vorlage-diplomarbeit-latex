\chapter{Viertes Kapitel}
\section{Listen und Aufz\"ahlungen}

Hier mal eine Auflistung von Elementen
\begin{itemize}
 \item erstes Element
 \item zweites Element
 \item noch ein Element
\end{itemize}

Hier mal eine Aufz\"ahlung
\begin{enumerate}
 \item erster Punkt
 \item noch ein Punkt
 \item letzter Punkt
\end{enumerate}

\section{Und n\"achster Abschnitt etwas l\"anger als vorher es war}
Eine neue Seite, um auchmal die Kopfzeile zu sehen, da sie auf Seiten mit Kapitelanfang nicht erscheinen. Eine Abk\"urzung ist z.B. etc.\abk{etc.}{et cetera}.

\section{Eine Tabelle}

Hier eine Tabelle:
\begin{table}[htbp]
\centering
\begin{tabular}{l|l|l|l}
SpalteA & SpalteB & SpalteC & SpalteD \\
\midrule
InhaltA1 & InhaltB1 & InhaltC1 & InhaltD1 \\
InhaltA2 & InhaltB2 & InhaltC2 & InhaltD2 \\
InhaltA3 & InhaltB3 & InhaltC3 & InhaltD3
\end{tabular}
\caption{Beispiel einer Tabelle}
\label{tab:tabelle1}
\end{table}

Wie man in der Tabelle~\ref{tab:tabelle1} sehen kann ...

% Neue Seite
\section{Zitieren und Literaturverzeichnis erzeugen}
Zitierregeln sind der Word-Version der Diplomarbeitsvorlage zu entnehmen! \\
In der Datei \textit{bib.bib} im Verzeichnis \textit{Literatur} sind neue Eintr\"age von Literatur hinzuzuf\"ugen. Das Format kann in der \LaTeX{} Dokumentation nachgesehen werden. \\
N\"utzliche Links zum automatischen Erstellen von BibTeX Eintr\"agen:
\begin{itemize}
	\item \url{http://truben.no/latex/bibtex/}
	\item \url{http://www.ottobib.com}
	\item \url{http://www.literatur-generator.de}
	\item \url{https://scholar.google.de}
\end{itemize}

\begin{center}
\Large{!!! \\Damit das Literaturverzeichnis erstellt wird, muss auch mit dem \textbf{Befehl BibTeX} kompiliert werden.\\ !!!}
\end{center}


\subsection{Beispiele}
Die exakte Herleitung kann in \cite{Prager1961} nachgelesen werden. \\
Hier ein Zitat etwas einger\"uckt:
\begin{quote}
Das Programm TeX wurde von Donald E. Knuth, Professor an der Stanford University, entwickelt. Leslie Lamport entwickelte Anfang der 1980er Jahre darauf aufbauend LaTeX, eine Sammlung von TeX-Makros. Der Name ist eine Abk\"urzung f\"ur Lamport TeX \cite{latextug}.
\end{quote}

\section{Bilder und Referenzen}
\begin{figure}[htbp]
\centering
\includegraphics[width=0.7\textwidth]{abbildungsdatei}
\caption{Titel der Abbildung} 
\label{fig:bild1}
\end{figure}

In der Abbildung~\ref{fig:bild1}\footnote{vgl. Zitat A\cite{referenzA}} % referenzA muss im Verzeichnis literatur in der Datei bib.bib enthalten sein
ist zu sehen, dass ...

\section{Formeln}
\label{sec:formeln}

\subsection{Albert Einstein}
Keine Formel der modernen Physik ist in der allgemeinen \"Offentlichkeit wohl so bekannt wie die Einsteinsche Formel:
\begin{equation*}
  E = mc^2
\end{equation*}.

\subsection{Allgemeine quadratische Gleichung}
\begin{equation*}
  ax^2 + bx + c = 0
\end{equation*}

\begin{equation*}
  x_{1/2} = \frac{-b \pm \sqrt{b^2-4ac}}{2a}
\end{equation*}

\subsection{Integral}
\begin{equation}
  \label{eq:1}
  \int\limits_{a}^{b} x^{2} \, dx = \frac{ b^{3} - a^{3} }{3}
\end{equation}
Siehe Beispiel \eqref{eq:1} \\
Siehe Abschnitt \ref{sec:formeln}

\section{Source Code einbinden}
\subsection{Datei}
Hier eine Einbindung von Source Code in Form einer Datei (funktioniert mit allen g\"angigen Programmiersprachen):
\lstinputlisting[label=lst:helloworldjava,caption=Hello World in Java]{./src/HelloWorld.java}
Die Ausgabe am Bildschirm von Hello World in Java ist in \ref{lst:helloworldjava} zu sehen.
\subsection{Inline}
Dieser Source Code ist direkt in \LaTeX{} eingegeben:
\begin{lstlisting}[caption=Hello World in C, label=lst:helloworldc]
#include <stdio.h>

/* Block
 * comment */
 
int main()
{
    // Line comment.
    printf("Hello World\n");
 
    return 0;
}
\end{lstlisting}
Die Ausgabe am Bildschirm von Hello World in Java ist in \ref{lst:helloworldc} zu sehen.