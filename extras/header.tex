%
% Paket fuer Uebersetzungen ins Deutsche
%
\usepackage[ngerman]{babel}

%
% Pakete um Latin1 Zeichnensaetze verwenden zu koennen und die dazu
% passenden Schriften.
%
\usepackage[latin1]{inputenc} % fuer OS X
% \usepackage[utf8]{inputenc} % fuer Windows
\usepackage[T1]{fontenc}

%
% Paket fuer Quotes
%
\usepackage[babel,french=guillemets,german=swiss]{csquotes}

%
% Paket zum Erweitern der Tabelleneigenschaften
%
\usepackage{array}

%
% Paket fuer schoenere Tabellen
%
\usepackage{booktabs}

%
% Paket um Grafiken einbetten zu koennen
%
\usepackage{graphicx}

%
% Spezielle Schrift im Koma-Script setzen.
%
\setkomafont{sectioning}{\normalfont\bfseries}
\setkomafont{captionlabel}{\normalfont\bfseries} 
\setkomafont{pagehead}{\normalfont\bfseries} % Kopfzeilenschrift
\setkomafont{descriptionlabel}{\normalfont\bfseries}

%
% Zeilenumbruch bei Bildbeschreibungen.
%
\setcapindent{1em}

%
% Kopf und Fu�zeilen
%
\usepackage{scrpage2}
\pagestyle{scrheadings}
% Inhalt bis Section rechts und Chapter links
\automark[section]{chapter}
% Mitte: leer
\chead{}

%
% mathematische symbole aus dem AMS Paket.
%
\usepackage{amsmath}
\usepackage{amssymb}

%
% Type 1 Fonts f�r bessere darstellung in PDF verwenden.
%
%\usepackage{mathptmx}           % Times + passende Mathefonts
%\usepackage[scaled=.92]{helvet} % skalierte Helvetica als \sfdefault
\usepackage{courier}            % Courier als \ttdefault

%
% Paket um Textteile drehen zu k�nnen
%
\usepackage{rotating}

%
% Paket fuer Farben im PDF
%
\usepackage{color}

%
% Paket fuer Links innerhalb des PDF Dokuments
%
\definecolor{LinkColor}{rgb}{0,0,0.5}
\usepackage[%
	pdftitle={Titel},% Titel der Diplomarbeit
	pdfauthor={Autor},% Autor(en)
	pdfcreator={LaTeX, LaTeX with hyperref and KOMA-Script},% Genutzte Programme
	pdfsubject={Betreff}, % Betreff
	pdfkeywords={Keywords}]{hyperref} % Keywords halt :-)
	
\hypersetup{colorlinks=true,% Definition der Links im PDF File
	linkcolor=LinkColor,%
	citecolor=LinkColor,%
	filecolor=LinkColor,%
	menucolor=LinkColor,%
	pagecolor=LinkColor,%
	urlcolor=LinkColor}

%
% Paket um LIstings sauber zu formatieren.
%
\usepackage[savemem]{listings}
\lstloadlanguages{TeX}

\renewcommand\lstlistlistingname{Quellcodeverzeichnis}
%
% Listing Definationen f�r PHP Code
%
\definecolor{lbcolor}{rgb}{0.85,0.85,0.85}
\lstset{language=[LaTeX]TeX,
	numbers=left,
	stepnumber=1,
	numbersep=5pt,
	numberstyle=\tiny,
	breaklines=true,
	captionpos=b,				% caption-position bottom
	breakautoindent=true,
	postbreak=\space,
	tabsize=2,
	basicstyle=\ttfamily\footnotesize,
	showspaces=false,
	showstringspaces=false,
	extendedchars=true,
	backgroundcolor=\color{lbcolor}}
%
% ---------------------------------------------------------------------------
%


%
% Neue Umgebungen
%
\newenvironment{ListChanges}%
	{\begin{list}{$\diamondsuit$}{}}%
	{\end{list}}

%
% aller Bilder werden im Unterverzeichnis figures gesucht:
%
\graphicspath{{bilder/}}

%
% Literaturverzeichnis-Stil
%
\bibliographystyle{plain}

%
% Anfuehrungsstriche mithilfe von \textss{-anzufuehrendes-}
%
\newcommand{\textss}[1]{"`#1"'}

%
% Strukturiertiefe bis subsubsection{} moeglich
%
\setcounter{secnumdepth}{3}

%
% Dargestellte Strukturiertiefe im Inhaltsverzeichnis
%
\setcounter{tocdepth}{3}

%
% Zeilenabstand wird um den Faktor 1.5 veraendert
%
%\renewcommand{\baselinestretch}{1.5}

%
% Abkuerzungsverzeichnis
%
\usepackage{nomencl}
% Befehl umbenennen in abk
\let\abk\nomenclature
% Deutsche Ueberschrift
\renewcommand{\nomname}{Abk\"urzungsverzeichnis}
% Punkte zw. Abkuerzung und Erklaerung
\setlength{\nomlabelwidth}{.20\hsize}
\renewcommand{\nomlabel}[1]{#1 \dotfill}
% Zeilenabstaende verkleinern
\setlength{\nomitemsep}{-\parsep}
\makenomenclature

\usepackage{eso-pic}
\newcommand\BackgroundPic{%
\put(0,0){%
\parbox[b][\paperheight]{\paperwidth}{%
\vfill
\centering
\includegraphics[width=0.8\paperwidth,height=0.8\paperheight,%
keepaspectratio]{bilder/titelbild.png}%
\vfill
}}}

%
% Kapitelueberschriften werden in eine Zeile geschrieben.
%
% normal: 
% Kapitel 1.
% Einleitung
%
% wenn hier auskommentiert:
% Kapitel 1: Einleitung
%
%\usepackage{titlesec}
%\titleformat{\chapter}[hang] 
%{\normalfont\huge\bfseries}{\chaptertitlename\ \thechapter:}{1em}{} 

