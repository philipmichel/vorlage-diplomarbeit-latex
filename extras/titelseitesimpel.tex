% Die Titelseite
% Im folgenden kommen ein paar Variablen, die auszuf�llen sind
% Bisher steht dort nur Musterinhalt
% Au�erdem m�ssen zei Dateien erstellt werden, Bild/Logo/Emblem des Fachgebietes
% sowie der Universit�t

\newcommand{\trtitle}{Titel der Diplomarbeit}
\newcommand{\trort}{Shkoder}
\newcommand{\trbetreuer}{Titel Betreuer}
\newcommand{\trdate}{\today}
\newcommand{\trnumber}{16.xx}
\newcommand{\trclass}{5X}
\newcommand{\trschuelereins}{Sch\"uler1}
\newcommand{\trschuelerzwei}{Sch\"uler2}
\newcommand{\trschuelerdrei}{Sch\"uler3}
\newcommand{\trschuelervier}{Sch\"uler4}
\newcommand{\trschuelerfuenf}{Sch\"uler5}
\newcommand{\trsauftraggeber}{Herr Max Mustermann oder Firma}
\newcommand{\trbetreuereins}{Lehrer1}
\newcommand{\trbetreuerzwei}{Lehrer2}
\newcommand{\trbetreuerdrei}{Lehrer3}

\thispagestyle{empty}

\begin{center}
  \large H\"ohere technische Schule f\"ur Informationstechnologie \\
  \large Shkolla e mesme profesionale private p\"er teknologji informacioni \\
  \Huge \"Osterreichiche Schule \glqq Peter Mahringer\grqq \\
  \Large Shkolla Austriake Shkod\"er
\end{center}

\begin{center}
  \includegraphics[scale=0.2]{htllogo} \\
  \textbf{\LARGE \trtitle}
\end{center}
\vspace{1cm}

\begin{flushleft}
\textbf{\LARGE Diplomarbeit Nr. \trnumber} \\
\LARGE Klasse \trclass{}, Schuljahr 2016/17

\begin{table}[htbp]
\Large
\begin{tabular}{cl}
   Ausgef\"uhrt von: & \trschuelereins \\ 
   & \trschuelerzwei \\ 
   & \trschuelerdrei \\ 
   & \trschuelervier \\ 
   & \trschuelerfuenf \\ 
 \end{tabular}
\end{table}
\end{flushleft}


\large Projektbetreuer1: \trbetreuereins \\
\large Projektbetreuer2: \trbetreuerzwei \\
\large Projektbetreuer3: \trbetreuerdrei \\
\newline
\large \trort{}, \today

\vfill